How we have read in previous Chapters, it has been possible to create a software for the extraction of three-dimensional models from medical images. In particular, we have seen how the combination between the topological approach and the High Performance Computing it has made possible the definition of extremely detailed three-dimensional models (as in Chapters \ref{Chapter41}, \ref{Chapter42} and \ref{Chapter43}). Moreover, we have seen hot this reconstruction is topologically correct, returning us the true shape of the object instead of its simplified representation. This advantage become evident in Chapter~\ref{Chapter42}, where we can compare the models obtained from marching cubes algorithm with the models obtained with the extraction of the boundary chain. In the first case it is not possible to see the interior of a single neuronal vein because the topology is not correct, in the other one it is possible.\\

However there are some open questions. In fact, until now, we have studied how to create in a efficient manner these models but we have not said nothing about visualization. In fact, in the previous examples the sizes of the final obj files were quite big, causing several problems. We need computer with a lot of memory space and the visualization is still slow. A solution can be found using new streaming algorithms which can adapt the output to the computer processing capabilities and to the detail level which is requested at that time.\\ 

Finally, although the results are quite appreciable (especially for the memory occupation), execution times could be further improved. In particular, the smoothing algorithm is quite slow. From an in-depth analysis, this fact depends on the search for adjacent vertices, which is done in linear time but is affected by the huge volume of data. A solution to this problem, is given from the adoption of an optimized pipeline conversion which can use less data at one time.