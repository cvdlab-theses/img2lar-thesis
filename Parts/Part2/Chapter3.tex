In this work the \textbf{Julia language} has been used as the main programming language. As we can see from its website\footnote{http://julialang.org/}, it is a high-level and high-performance language for technical computing, with a syntax that is well known for whom uses scientific programming languages. It also provides a compiler, distributed parallel execution and a mathematical function library. We chose it because of these characteristics and because it permits to parallelize programs in a very simple way.\\

In the next sections we will see in detail its characteristics in the detail and the code used for parallel computation.

\section{Principal characteristics}\label{sec23:julia}
As we can see in the Julia website, this language has the following characteristics:
\begin{itemize}
 \item Multiple dispatch: providing ability to define function behavior across many combinations of argument types
 \item Dynamic type system: types for documentation, optimization, and dispatch
 \item Good performance, approaching statically-compiled languages like C
 \item Built-in package manager
 \item Lisp-like macros and other metaprogramming facilities
 \item Call Python functions: use of the PyCall package
 \item Call C functions directly: no wrappers or special APIs
 \item Powerful shell-like capabilities for managing other processes
 \item Designed for parallelism and distributed computation
 \item Coroutines: lightweight “green” threading
 \item User-defined types are as fast and compact as built-ins
 \item Automatic generation of efficient, specialized code for different argument types
 \item Elegant and extensible conversions and promotions for numeric and other types
 \item Efficient support for Unicode, including but not limited to UTF-8
 \item MIT licensed: free and open source
\end{itemize}

From the syntax point of view, Julia takes inspiration from various dialects of \textit{Lisp}, including \textit{Scheme} and \textit{Common Lisp}, and it shares many features with \textit{Dylan} and \textit{Fortress}. It is also possible to implement \textbf{metaprogramming} using macros. They are necessary because they are executed when code is parsed, therefore, macros allow the programmer to generate and include fragments of customized code before the full program is run. We can see the difference in the following example taken from the official documentation:\\

\begin{julia}
 > macro twostep(arg)
     println("I execute at parse time. The argument is: ", arg)
     return :(println("I execute at runtime. The argument is: ", $arg))
   end
 
 > ex = macroexpand( :(@twostep :(1, 2, 3)) );
   "I execute at parse time. The argument is: :((1,2,3))"
 
 > typeof(ex)
   Expr

 > ex
   :(println("I execute at runtime. The argument is: ",$(Expr(:copyast, :(:((1,2,3)))))))

 > eval(ex)
   "I execute at runtime. The argument is: (1,2,3)"
\end{julia}

Macros are invoked using the following syntax:
\begin{julia}
 @name expr1 expr2 ...
\end{julia}

In addition, Julia includes an interactive session shell called \textbf{REPL}, which can be used to make experiments and fast code testing. For example we can write:
\begin{julia}
 julia> p(x) = 2x^2 + 1; f(x, y) = 1 + 2p(x)y
 julia> f(0, 4)
 9
\end{julia}

All these commands can also be written in a script file with \texttt{.jl} extension and executed in the shell with:
\begin{julia}
 user@pc: julia <filename>
\end{julia}

As seen in the above list, Julia has good performances, which are achieved using \textbf{just-in-time (JIT)} compilation, implemented using \textit{LLVM}. This compiler, combined with the language's design allow it to approach the C performances. In table~\ref{tbl:JuliaBenchmark} we can see a little benchmark coming from the official website:

\begin{table}[htb]
\centering
\caption[benchmark times relative to C]{benchmark times relative to C (smaller is better, C performance = 1.0). Table taken from the official documentation}
\label{tbl:JuliaBenchmark}
\begin{tabular}{p{2cm} | p{1.2cm} | p{1.3cm} | p{1.3cm} | p{1.3cm} | p{1.2cm} | p{1.2cm} | p{1.2cm}}
\toprule
&\textbf{fib}& \textbf{parse-int}&\textbf{quick-sort}& \textbf{mandel}&\textbf{pi-sum}& \textbf{rand-mat-stat}& \textbf{rand-mat-mul}\\ \midrule
\textbf{Fortran} 	& 0.70	& 5.05	& 1.31	& 0.81	& 1.00	& 1.45	& 3.48\\ \midrule
\textbf{Julia}		& 2.11	& 1.45	& 1.15	& 0.79	& 1.00	& 1.66	& 1.02\\ \midrule
\textbf{Python}         & 77.76	& 17.02	& 32.89	& 15.32	& 21.99	& 17.93	& 1.14\\ \midrule
\textbf{R}		&533.52	& 45.73	&264.54	& 53.16	& 9.56	& 14.56	& 1.57\\ \midrule
\textbf{Matlab}		& 26.89	&802.52	& 4.92	& 7.58	& 1.00	& 14.52	& 1.12\\ \midrule
\textbf{Octave}         &9324.35&9581.44&1866.01&451.81	&299.31	& 30.93	& 1.12\\ \midrule
\textbf{Mathematica}	&118.53	&15.02	& 43.23	& 5.13	& 1.69	& 5.95	& 1.30\\ \midrule
\textbf{Javascript}	& 3.36	& 6.06	& 2.70	& 0.66	& 1.01	& 2.30	& 15.07\\ \midrule
\textbf{Go}		& 1.86	& 1.20	& 1.29	& 1.11	& 1.00	& 2.96	& 1.42\\ \midrule
\textbf{LuaJIT}         & 1.71	& 5.77	& 2.03	& 0.67	& 1.00	& 3.27	& 3.27\\ \midrule
\textbf{Java}	        & 1.21	& 3.35	& 2.60	& 1.35	& 1.00	& 3.92	& 2.36\\ \midrule
\bottomrule
\end{tabular}
\end{table}

Julia also supports modular applications. In particular, modules in Julia are separate global variable workspaces. They are delimited by the keywords:
\begin{julia}
 module Name
 ...
 end
\end{julia}

With modules, one can create top-level definitions without worrying about name conflicts. Within a module, it is possible to control which names from other modules are visible (via importing), and to specify which names are intended to be public (via exporting). In the official documentation we can find the following example:
\begin{julia}
 module MyModule
 export x, y
 x() = "x"
 y() = "y"
 p() = "p"
end
\end{julia}
In this module we export the x and y functions (with the keyword \texttt{export}), and also have the non-exported function p. In Table~\ref{tbl:juliaImports} there are several different ways to load the Module and its inner functions into the current workspace.

\begin{table}[htb]
\centering
\caption[Modules loading in Julia]{Modules loading in Julia. Table taken from the official documentation}
\label{tbl:juliaImports}
\begin{tabular}{p{5cm} | p{7cm}}
\toprule
\textbf{Import command}			  & \textbf{What is brought into scope}\\ \midrule
\texttt{using MyModule}			  & All exported names (x and y), MyModule.x, MyModule.y and MyModule.p\\ \midrule
\texttt{using MyModule.x, MyModule.p}     & x and p\\ \midrule
\texttt{using MyModule: x, p}          	  & x and p\\ \midrule
\texttt{import MyModule}	          & MyModule.x, MyModule.y and MyModule.p\\ \midrule
\texttt{import MyModule.x, MyModule.p}	  & x and p\\ \midrule
\texttt{import MyModule: x, p}            & x and p\\ \midrule
\texttt{importall MyModule}               & All exported names (x and y)\\ \midrule
\bottomrule
\end{tabular}
\end{table}

Obviously, we can use file scripts for defining multiple modules. Moreover, including the same code in different modules provides mixin-like behavior. One could use this to run the same code with different base definitions, for example testing code by running it with “safe” versions of some operators:
\begin{julia}
module Normal
include("mycode.jl")
end

module Testing
include("safe_operators.jl")
include("mycode.jl")
end
\end{julia}

\section{Parallel programming in Julia}\label{sec23:parallelJulia}

How we have seen in the earlier chapter, modern computers possesses more than one CPU and several computers can be combined together in a cluster. Moreover we have seen the great advantages that this type of architecture can return. For its parallel features, Julia has chosen an environment based on \textit{message passing} (so we have multiple processes which run in separate memory domains). However, the Julia's implementation of message passing is a bit different from other environments such as MPI. In fact communication is generally "one-sided", so the programmer needs to explicitly manage only one process in a two-process operation. Furthermore these operations do not seem "message send" and "message receive" but resemble higher-level operations like calls to user functions.\\

The most important primitives are \textbf{remote references} and \textbf{remote calls}. The former is an object that can be used from any process to refer to an object stored on a particular process, while the latter is a request by one process to call a certain function on certain arguments on another process. A remote call returns a remote reference to its result and \textit{returns immediately}, while its process proceeds to the next operation. It is possible to wait for a remote call to finish by calling the function \texttt{wait} on its remote reference (or \texttt{fetch} if we are waiting for its value).\\

Now we can examine some useful functions for parallel programming. The simpler function is \texttt{remotecall()}, which takes as parameters the \textit{index of the process that will do the work}, the function that will be invoked and its parameters. It is considered a low-level interface that provides a finer control on the software. However, most parallel programming in Julia does not reference specific processes or the number of processes available. So usually Julia programmers use the \texttt{@spawn} macro, which operates on an expression, and choose where to do the operation. For example we can write:
\begin{julia}
 julia> r = @spawn rand(2,2)
 RemoteRef(1,1,0)

 julia> s = @spawn 1 .+ fetch(r)
 RemoteRef(1,1,1)

 julia> fetch(s)
 1.10824216411304866 1.13798233877923116
 1.12376292706355074 1.18750497916607167
\end{julia}
Note the use of \texttt{1.+fetch(r)} instead of \texttt{1.+r}. This is because we do not know where the code will run, so in general a \texttt{fetch()} might be required to move r to the process doing the addition. In this case, \texttt{@spawn} is smart enough to perform the computation on the process that owns r, so the \texttt{fetch()} will be a no-op.

Now we have to do a final note on code availability. In fact, according to Julia's parallel architecture, the code must be available on any process that runs it. For example we can observe this code from the official documentation:
\begin{julia}
 julia> function rand2(dims...)
          return 2*rand(dims...)
        end

 julia> rand2(2,2)
 2x2 Float64 Array:
  0.153756  0.368514
  1.15119   0.918912

 julia> @spawn rand2(2,2)
 RemoteRef(1,1,1)
 julia> @spawn rand2(2,2)
 RemoteRef(2,1,2)
 julia> exception on 2: in anonymous: rand2 not defined
\end{julia}

As as consequence, we have to be aware of this fact when loading code from modules. Generally speaking, we can have the following cases:
\begin{itemize}
 \item \texttt{include("ModuleName.jl")} loads the file on just a single process
 \item \texttt{using ModuleName} causes the module to be loaded on all processes; however, the module is brought into scope only on the one executing the statement
\end{itemize}
In addition, we can force a command to run on all processes using the \texttt{@everywhere} macro. For example we can write:
\begin{julia}
 @everywhere using ModuleName
\end{julia}

So far we have seen the most interesting features of the Julia language (in particular the ones involved in parallel computing). At this point, the reader should have understood the reasons within the choice of this language and the decisions at the basis of the developed library